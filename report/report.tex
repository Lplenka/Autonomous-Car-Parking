\documentclass{svproc}
\usepackage{url}
\def\UrlFont{\rmfamily}

\begin{document}
\mainmatter
\title{Autonomous Car Parking
}
\subtitle{CS7IS2 Project (2021/2022)}
\author{Mayuresh Shelke, Ming Jun Lim, Lalu Prasad Lenka and Samridh James}

\institute{
\email{shelkem@tcd.ie}, \email{limm5@tcd.ie}, \email{lplenka@tcd.ie}, \email{jamessa@tcd.ie}}

\maketitle              % typeset the title of the contribution

\begin{abstract}
%The abstract should summarize the contents of the report and should contain at least 70 and at most 150 words. 
%The abstract should give a concise overview of the main points of the report: the motivation behind the work, a very high level description of the problem and how it was solved by the proposed algorithms. The abstract must not include any figures or table.
Motivation behind the work, high level description of the problem, how it was solved by the proposed algorithms.
% TODO: Motivation behind the work
% TODO: High level description of the problem
% TODO: How it was solved by the proposed algorithms
\keywords{soft-actor-critic, behaviour cloning, evolutionary algorithm, parking}
\end{abstract}
%


%The report will be graded on the basis of:
%
%\begin{itemize}
%\item Originality - 10\%;
%\item Technical soundness - 20\%;
%\item Organisation - 20\%;
%\item Clarity of presentation - 20\%;
%\item Adequacy of bibliography/Results - 10\%
%\item Presentation slides and recording) - 20\%
%\end{itemize}


%\begin{description}
%Your report should provide a survey and an experimental comparison of multiple solution approaches to a particular problem. This is a critical review of at least three papers that significantly contributed to advance the state-of-the-art for the problem you are analysing. It should not be a mere summary of the papers. You are expected to conduct an analytical review of the methods under analysis to try to find common aspect and differences, connections between methods, drawbacks and open problems. Unless the faced problem has emerged recently, students should choose their papers by diversifying the range of approaches used to solve the problem. A good guideline could be to choose a paper from a decade or two ago, and a couple of more recent papers. You need to experimentally evaluate approaches in a simulation of a problem, in a range of scenarios, and analyse the pros and cons of each approach. 
%\end{description}

\section{Introduction}
\label{sec:introduction}
%In this section, you should introduce your work: what are the motivations behind this work? What is the relevant problem that you are investigating? Why is it relevant? 
%Briefly, introduce the background information required to understand the problem and the concepts that you will develop. 
%
%This section should also contain the link to the recording of your presentation (college OneDrive link – please make sure sharing permissions are such that everyone with tcd email can access it)
%
% TODO: Motivations behind this work
The Autonomous Vehicle (AV) industry is one of the most active and promising area in the recent years with many ongoing research to make it fully autonomous \cite{talavera2021autonomous}. The International Society of Automotive Engineers (SAE) proposed a six-degree autonomy scale with level 0 as no automation and level 5 as full automation without any human interventions \cite{stoma2021future}. The current level of autonomy scale achieved today is between level 2 and 3 requiring some assistance from the driver and limited to ideal conditions.

% TODO: What is the relevant problem that you are investigating
Car parking is challenging and disliked by most drivers due to time required searching for spaces, risk of scratching vehicle, safety of pedestrian, etc \cite{baburaj2021smart}\cite{bosch_global_2022}. It requires the driver to estimate the space required for the car and manoeuvre it into available space by controlling the steering angle and accelerator. While in fact the traffic report statistics survery results found in 12 million traffic accidents, there are about 10,000 traffic accidents occurring in the parking lots with many more number of accidents not reported \cite{wang2014automatic}.

% TODO: Why is it relevant
Most vehicles are parked $95\%$ throughout the lifetime \cite{choi2019self}. Parking is required by every driver and apply AV will greatly improve the quality of life and ease of drivers. It increases driving safety by utilising various sensors to understand the environment surround vehicle and park at the designated space avoid collisions due to lack of human experiences \cite{wang2014automatic}. Other potential application for autonomous car parking is saving spaces in car parking lots by parking in an optimal grid resulting in a much efficient parking lot capable of more car \cite{choi2019self}.

% TODO: Introduce background information required to understand the problem and concepts that you will developed
The highway-env github repository\footnote{https://github.com/eleurent/highway-env} contains a collection of autonomous driving and tactical decision-making tasks environment \cite{highway-env}. The parking environment\footnote{id of "parking-v0"} is selected for the project which is a goal-conditioned continuous control task in a given space with a vehicle aiming to reach the destination point. The parking environment uses a grid like layout similar to actual car parks. the vehicle requires input of the steering angle and accelerator. The selected algorithms to be investigated in the project are Soft-Actor-Critic (SAC) from the reinforcement learning family, behaviour cloning from the imitation family and evolutionary algorithm from the genetic population-based family.
% TODO: Include link to recording of presentation

\section{Related Work}
% In this section you will discuss possible approaches to solve the problem you are addressing, 
% justifying your choice of the 3 you have selected to evaluate. 
% Also, briefly introduce the approaches you are evaluating with a specific emphasis on differences and similarities to the proposed approach(es).
The autonomous vehicle control system consists of mission trajectory planning system generating path for vehicle to reach target, navigation system detecting environment and either moving or fixed target and trajectory tracking controller to navigate the vehicle accordingly \cite{lin2018path}. Path planning problem is the main subject of most autonomous vehicle studies. Prior to path planning, the algorithm needs to know the state of the vehicle such as its position and direction. GPS is commonly used to deduce the state by using signals Line Of Sight (LOS) from positioning satellite but not possible in the context of indoor parking lots \cite{correa2017autonomous} and high-rise buildings \cite{saksena2019towards}. Instead other data be used to monitor the state of vehicle such as vision-based sensors and LiDAR sensors \cite{chan2021review}. The works related to autonomous car parking uses different inputs and types algorithm families, this section will mainly focus on selected algorithms in Section \ref{sec:introduction}.

% TODO: Soft-Actor-Critic (SAC) related works

% TODO: Behaviour Cloning related works
Behaviour Cloning (BC) is from the Imitation learning family which uses observations from the expert to learn \cite{torabi2018behavioral}. In paper \cite{saksena2019towards} trains a Recurrent Convolutional Network for autonomous driving in lane changing context by using BC to learn from human experts driving vehicles. The paper noted the ease of changing architecture when using BC technique and found that spatio-temporal properties worked well in dynamic environments.
%The paper focuses more on spatio-temporal perception property from the network architecture to work accordingly in the dynamic environments and ease of changing the architecture when using BC technique.
%The paper noted that BC model has poor generalisation ability and suffers in unseen scenarios. 
%It uses BC as a means of extracting information from the raw data and an input for training the proposed network architecture.

% TODO: Evolutionary algorithm related works

\section{Problem Definition and Algorithm}
Problem definition and algorithm
%This section formalises the problem you are addressing and the models used to solve it. This section should provide a technical discussion of the chosen/implemented algorithms. A pseudocode description of the algorithm(s) can also be beneficial to a clear explanation. It is also possible to provide one example that clarifies the way an algorithm works. It is important to highlight in this section the possible parameters involved in the model and their impact, as well as all the implementation choices that can impact the algorithm.

%\subsection{Subsection Title}

\section{Experimental Results}
Experimental result
%This section should provide the details of the evaluation. Specifically:
%\begin{itemize}
%\item Methodology: describe the evaluation criteria, the data used during the evaluation, and the methodology followed to perform the evaluation. 
%\item Results: present the results of the experimental evaluation. Graphical data and tables are two common ways to present the results. Also, a comparison with a baseline should be provided.
%\item Discussion: discuss the implication of the results of the proposed algorithms/models. What are the weakness/strengths of the method(s) compared with the other methods/baseline?
%\end{itemize}

\section{Conclusions}
Conclusion
%Provide a final discussion of the main results and conclusions of the report. Comment on the lesson learnt and possible improvements.
%
%
%A standard and well formatted bibliography of papers cited in the report. For example:

\bibliographystyle{plain} % We choose the "plain" reference style
\bibliography{refs} % Entries are in the refs.bib file
%\begin{thebibliography}{6}
%%
%
%\bibitem {smit:wat}
%Smith, T.F., Waterman, M.S.: Identification of common molecular subsequences.
%J. Mol. Biol. 147, 195?197 (1981). \url{doi:10.1016/0022-2836(81)90087-5}
%
%\bibitem {may:ehr:stein}
%May, P., Ehrlich, H.-C., Steinke, T.: ZIB structure prediction pipeline:
%composing a complex biological workflow through web services.
%In: Nagel, W.E., Walter, W.V., Lehner, W. (eds.) Euro-Par 2006.
%LNCS, vol. 4128, pp. 1148?1158. Springer, Heidelberg (2006).
%\url{doi:10.1007/11823285_121}
%
%\bibitem {fost:kes}
%Foster, I., Kesselman, C.: The Grid: Blueprint for a New Computing Infrastructure.
%Morgan Kaufmann, San Francisco (1999)
%
%\bibitem {czaj:fitz}
%Czajkowski, K., Fitzgerald, S., Foster, I., Kesselman, C.: Grid information services
%for distributed resource sharing. In: 10th IEEE International Symposium
%on High Performance Distributed Computing, pp. 181?184. IEEE Press, New York (2001).
%\url{doi: 10.1109/HPDC.2001.945188}
%
%\bibitem {fo:kes:nic:tue}
%Foster, I., Kesselman, C., Nick, J., Tuecke, S.: The physiology of the grid: an open grid services architecture for distributed systems integration. Technical report, Global Grid
%Forum (2002)
%
%\bibitem {onlyurl}
%National Center for Biotechnology Information. \url{http://www.ncbi.nlm.nih.gov}
%\end{thebibliography}
\end{document}
